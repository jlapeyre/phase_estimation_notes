%% Force arXiv to use pdflatex
\pdfoutput=1
\documentclass[superscriptaddress,nofootinbib,longbibliography,aps,pre]{revtex4-1}

\usepackage{graphicx}  %% figures
\usepackage{amssymb}   %% mathbb
\usepackage{amsmath}   %% align and other math environments
\usepackage[usenames,dvipsnames]{color}


\graphicspath{{figures/}}


\usepackage{xcolor} %% needed for frame color
\usepackage{mdframed}
%\mdfsetup{backgroundcolor=gray,shadow=true,roundcorner=8pt}
\mdfsetup{
outerlinewidth = 2 ,%
roundcorner = 20 pt ,%
leftmargin = 0 ,%
rightmargin = 0 ,%
backgroundcolor = brown!20, %
outerlinecolor = blue!70!black,%
%innertopmargin = 0 ,%  % Note that this is changed to
innertopmargin = 5 ,  % For two column mode, else boxes are two big
innerbottommargin = 5,%
splittopskip = 10 ,%
}

\usepackage[]{hyperref} % Make much nicer links.
\hypersetup{
  colorlinks   = true, %Coloured links instead of ugly boxes
  urlcolor     = blue, %Colour for external hyperlinks
  linkcolor    = blue, %Colour of internal links
  citecolor   = red,    %Colour of citations
%  hypertexnames = false % for autonum
}

\newcommand{\myhead}[1]{\textit{#1} --- \ }


\newcommand{\adj}{\dagger}
\newcommand{\tr}{{\operatorname{tr}}}

\newcommand{\ket}[1]{|{#1}\rangle}
\newcommand{\bra}[1]{\langle{#1}|}
\newcommand{\braket}[2]{\langle{#1}|{#2}\rangle}
\newcommand{\ketbra}[2]{|{#1}\rangle\!\langle{#2}|}

% Set of complex numbers
\newcommand{\Cn}{{\mathbb{C}}}

% m x n complex matrices ("r" for rectangular)
\newcommand{\Cmatr}[2]{{\Cn}^{#1\times #2}}

% n x n complex matrices
\newcommand{\Cmat}[1]{\Cmatr{#1}{#1}}

% 2 x 2 complex matrices
\newcommand{\Cmatt}{\Cmat{2}}

\newcommand{\umax}{u_{\text{max}}}
\newcommand{\umin}{u_{\text{min}}}

\newcommand{\aphi}{\tilde{\phi}}
\newcommand{\auvec}{\tilde{u}}

\begin{document}

\title{Phase estimation implementation}
\author{G. John \surname{Lapeyre}, Jr.}
\date{\today}

{
\let\clearpage\relax
\maketitle
}

\section{Scaling the Hamiltonian for quantum phase estimation}

Consider an eigensystem
%
\begin{equation}
  H\ket{u_{j}} = u_{j}\ket{u_{j}}
\end{equation}
%
We choose $\tau>0$ so that upon rescaling $H' = -\tau H$, the unitary describing evolution is
%
\begin{equation}
  U = e^{-i H'} = e^{i\tau H}.
\end{equation}
%
Then the phase imparted is determined by
%
\begin{equation}
  U\ket{u_{j}} = e^{i2\pi\phi_{j}} \ket{u_{j}}, \ \text{ with } \phi_{j}=\tau \frac{u_{j}}{2\pi}.
\end{equation}
%
Recall that $e^{i2\pi(\phi_{j} + m)}$ has the same value for all integers $m$.
The phase approximation algorithm provides an estimate
$\aphi$ of $\phi_{j} \!\!\mod 1$.
%Physically, we cannot distinguish among $\phi + m2\pi$ for different values of $m$.
We would like to choose $\tau$ such that the phase satisfies $0\le \phi_{j} < 1$,
which then guarantees a one-to-one correspondence between $u_{j}$ and $\phi_{j}$.

\subsection{A single eigenstate and positive spectrum}

Suppose the spectrum is positive
%
\begin{equation}
  u_{j} \ge 0, \ \text{ for all } j.
\end{equation}
%
The algorithm ideally measures the phase as $\aphi = \phi_{j}\mod 1$.
So, we choose a bound $u_{b} > \umax$ where  $\umax = \max_{j} u_{j}$
so that $\tau = 2\pi/u_{b}$, and $\phi_{j} = u_{j}/u_{b}$ satisfies $0\le \phi_{j}<1$.
Then there is a one-to-one correspondence between values
of $\phi$ and $u_{j}$.

In summary, we supply to the algorithm
an eigenstate $\ket{u_{j}}$ and a unitary
constructed\footnote{%
  On quantum hardware, we might construct $-H$ and evolve for time $\tau = 2\pi/u_{b}$.
 }
from
%
\begin{equation}
  -\tau H = -\frac{2\pi}{u_{b}} H.
\end{equation}
%
We read the output $\aphi$ and obtain our estimate of $\tilde{u_{j}}$ of $u_{j}$ as
%
\begin{equation}
  \tilde{u_{j}} = u_{b} \aphi.
\end{equation}
%

In fact, the phase is read from a register with a finite number $n$ of bits and so
takes discrete values
%
\begin{equation}
  \label{phaseregvals}
  \aphi \in \{0, \epsilon, 2\epsilon, 3\epsilon, \ldots, 1 - \epsilon\}, \ \text{ where } \epsilon = 2^{-n}.
\end{equation}
%
The estimate $\tilde{u_{j}}$ takes values
%
\begin{equation}
  \label{eigvaldiscrete}
  \tilde{u_{j}} \in \{0, \epsilon_{u}, 2\epsilon_{u}, 3\epsilon_{u}, \ldots, u_{b} - \epsilon_{u}\},
  \ \text{ where } \epsilon_{u} = u_{b}2^{-n}.
\end{equation}
%
In particular, if we choose $u_{b}=\umax$, then $\umax$ is mapped to zero.
This is why we required $u_{b} > \umax$  rather than $u_{b} \ge \umax$.


\subsubsection{Example: Exact representation of minimum eigenvalue}

When testing implementations, one may encounter problems such as the following.
Still assuming $u_{j}>0$ for all $j$, we look for a scaling that allows at the same time
%
\begin{enumerate}
   \item{an exact representation of $\umin$ in the $n$ bits of the phase register.}
   \item{$u_{b} > \umax$}.
\end{enumerate}
%
The second condition is required if we expect that a component $\ket{\umax}$ will be present in the input state.
Referring to~\eqref{eigvaldiscrete} we see that for the first condition we must have $\umin = m\epsilon_{u}$ or
%
\begin{equation}
  u_{b} = \frac{2^{n}}{m} \umin, \ \text{ for some } m \in {1,2,\ldots, n-1}.
\end{equation}
%
Note that we do not include $m=0$ as a possibility.
We only scale the phase interval, but do not shift it.
In particular, the phase $\phi=0$ corresponds only to $u=0$
(unless $\umax >= u_{b}$, which we want to avoid.)
The second condition, $u_{b} > \umax$ may be written
%
\begin{equation}
  \label{umaxumincond}
  \umax < \frac{2^{n}}{m} \umin
\end{equation}
%
Clearly if~\eqref{umaxumincond} is satisfied for any $m>1$, then it
is satisfied for $m=1$.
That is $m=1$ gives the largest $u_{b}$.
Thus, setting $m=1$, we find the condition for the number of qubits in the phase register $n$.
%
\begin{equation}
  n > \log_{2}\left(\frac{\umax}{\umin} \right)
\end{equation}
%
This is the number of qubits necessary to satisfy both conditions 1 and 2 above.
On the other hand, to get the best resolution, we want to choose $m$ as large
as possible. From~\eqref{umaxumincond} we find that for fixed $n$, we can
choose the largest $m$ satisfying
%
\begin{equation}
  m < 2^{n}\frac{\umax}{\umin},
\end{equation}
%
Explicitly,
%
\begin{equation}
  m =  \left \lfloor{2^{n}\frac{\umax}{\umin}} \right \rfloor,
\end{equation}
%

\subsection{Single eigenstate and positive and negative eigenvalues}

Here we present a straigtforward scaling that makes good use of the resolution
in case the spectrum has both positive and negative values of the same magnitude.
We map the positive eigenvalues to $(0, \pi)$
and the negative eigenvalues to $(-\pi, 0)$.
We choose $u_{b}$ satisfying
%
\begin{equation}
  u_{b} > \max \left\{ |\umin|, |\umax| \right\}.
\end{equation}
%
For a positive spectrum we chose $\tau=2\pi/u_{b}$.
Here, we choose $\tau=\pi/u_{b}$, so that
$\phi_{j}=u_{j}/(2u_{b})$ satisfies
%
\begin{equation}
-\frac{1}{2} \le \phi_{j} < \frac{1}{2}.
\end{equation}
%
We obtain the unwrapped observed phase $\aphi'$
satisfying $-(1/2) \le \aphi' < (1/2)$ via
%
\begin{equation}
  \aphi' =
  \begin{cases}
    \aphi & \text{ if } \aphi < \frac{1}{2} \\
    1 - \aphi & \text{ if } \aphi \ge \frac{1}{2} \\
  \end{cases}.
\end{equation}
%
The sorted possible values of $\aphi'$ are given by
%
\begin{equation}
  \aphi' \in \left\{-\frac{1}{2}, -\frac{1}{2} + \epsilon, -\frac{1}{2} +  2\epsilon, \ldots, \frac{1}{2} - \epsilon \right\},
  \ \text{ where } \epsilon = 2^{-n}.
\end{equation}
%
The estimated eigenvalue, obtained via $\auvec = 2u_{b}\aphi'$, has value given by
%
\begin{equation}
  \auvec \in \left\{-u_{b}, -u_{b} + \epsilon_{u}, \ldots, u_{b} - \epsilon_{u} \right\},
  \ \text{ where } \epsilon_{u} = u_{b}2^{-n+1}.
\end{equation}
%

\section{Fast simulation of the phase register probabilities}

Here, we present a method to compute numerically the probabilities
of the states of phase-estimation register in parallel QPE.
The method is very straightforward.
We simply compute intermediate steps classically rather than with quantum circuits.
Recall that QPE works with a composite system with Hilbert space $V_{p}\otimes V_{U}$
where $V_{p}$ corresponds to an $n$-qubit phase-estimation register and $V_{U}$ is
the space of the eigenphase problem.
Suppose for the moment that the input state $\ket{u}\in V_{U}$ is an eigenvector of
$U$, that is $U\ket{u}=e^{2\pi i \phi_{u}}\ket{u}$.
For our purposes, we break QPE into three steps.
%
\begin{enumerate}
\item{Preparation of a state
    \begin{equation}
      \label{aftercontrolledU}
      \frac{1}{\sqrt{N}} \sum_{j=0}^{N-1} e^{2\pi i j \phi_{u}} \ket{j}\ket{u},
    \end{equation}
    where $N=2^{n}$ and the states $\ket{j}$ are the computational basis vectors of the phase register.
  }
\item{Application of the inverse (quantum) Fourier transform of the prepared state, with the result
    \begin{equation}
      \ket{\tilde{\phi}_{u}} \ket{u}.
    \end{equation}
    If $\phi_{u}$ can be representated exactly in $n$-qubits, then $\phi_{u}=k/N$ for some integer $k$.
    In this case we are performing the inverse Fourier transform of a Fourier transform with a single frequency
    component, given by $k$. In this case, the phase estimation register will be in the computational basis state $\ket{k}$.
    }
\item{Measurement of the state $\ket{\tilde{\phi}_{u}}$ in the computational basis to obtain an estimate of $\phi_{u}$}
\end{enumerate}
%
Our procedure for computing the amplitudes of $\ket{\tilde{\phi}_{u}}$
takes as input the phase $\phi_{u}$, and the number of qubits $n$.
In particular, we make no use of $\ket{u}$ and $U$.
We first prepare complex floating point vector $x_{0},\ldots, x_{N-1}$ with values
%
\begin{equation}
  x_{j} = \frac{1}{\sqrt{N}} \sum_{j=0}^{N-1} e^{2\pi i j \phi_{u}}.
\end{equation}
%
We then compute the inverse Fourier transform,
 via the FFT, of the vector $x_{0},\ldots, x_{N-1}$.
The result is a numerical approximation of amplitudes of $\ket{\tilde{\phi}_{u}}$
in the computational basis, that is $\braket{j}{\tilde\phi_{u}}$ for $j=0,\ldots, N-1$.
These are amplitudes of the phases in~\eqref{phaseregvals}
% by taking the discrete inverse Fourier
% transform,
% The result is a vector of the amplitudes of the values in
% given by $\braket{j}{\tilde\phi_{u}}$ for $j=0,\ldots, N-1$.

Suppose that instead of an eigenstate $\ket{u}$ we have as input a generic pure state
$\sum_{u}c_{u}\ket{u}$, where the sum runs over eigenvectors of $U$.
In this case, the QPE algorithm leaves the phase register in the state
%
\begin{equation}
 \sum_{u} c_{u} \ket{\tilde\phi_{u}}\ket{u}.
\end{equation}
%
The state of the phase register alone is described by the reduced density matrix
%
\begin{equation}
 \rho_{\text{phase}}  = \sum_{u} |c_{u}|^{2} \ketbra{\tilde\phi_{u}}{\tilde\phi_{u}}.
\end{equation}
%
The probability of reading the computational basis state $\ket{j}$ is then
%
\begin{equation}
  p(j) = \tr\left(\ketbra{j}{j}\rho_{\text{phase}}\right)
   = \sum_{u} |c_{u}|^{2} |\braket{j}{\tilde\phi_{u}}|^{2}.
\end{equation}
%
Consider taking as input the number of qubits in the phase register $n$
and pairs $(\phi_{u}, c_{u})$ representing the input state.
We can compute $p(j)$ numerically as follows.
Initialize $p(j)$ to zero for all $j$.
Loop over $u$.
For each iteration with fixed $u$,
compute the vector of $\braket{j}{\tilde\phi_{u}}$ for $j=0,\ldots, N-1$ as described above.
For each $j$ accumulate $|c_{u}|^{2} |\braket{j}{\tilde\phi_{u}}|^{2}$ into $p(j)$.
Continue to the next $u$ until finished.



\end{document}


%%% Local Variables:
%%% mode: latex
%%% TeX-master: t
%%% End:
